\documentclass[12pt,]{article}
\usepackage[margin=1in]{geometry}
\newcommand*{\authorfont}{\fontfamily{phv}\selectfont}
\usepackage[]{libertine}
\usepackage{abstract}
\renewcommand{\abstractname}{}    % clear the title
\renewcommand{\absnamepos}{empty} % originally center
\newcommand{\blankline}{\quad\pagebreak[2]}

\usepackage[portuges]{babel}

\providecommand{\tightlist}{%
  \setlength{\itemsep}{0pt}\setlength{\parskip}{0pt}} 
\usepackage{longtable,booktabs}

\usepackage{parskip}
\usepackage{titlesec}
\titlespacing\section{0pt}{12pt plus 4pt minus 2pt}{6pt plus 2pt minus 2pt}
\titlespacing\subsection{0pt}{12pt plus 4pt minus 2pt}{6pt plus 2pt minus 2pt}

\titleformat*{\subsubsection}{\normalsize\itshape}

\usepackage{titling}
\setlength{\droptitle}{-.25cm}

%\setlength{\parindent}{0pt}
%\setlength{\parskip}{6pt plus 2pt minus 1pt}
%\setlength{\emergencystretch}{3em}  % prevent overfull lines 

\usepackage[T1]{fontenc}
\usepackage[utf8]{inputenc}

\usepackage{fancyhdr}
\pagestyle{fancy}
\usepackage{lastpage}
\renewcommand{\headrulewidth}{0.3pt}
\renewcommand{\footrulewidth}{0.0pt} 
\lhead{}
\chead{}
\rhead{\footnotesize Economia Política e Instituições -- Março-Maio de 2017}
\lfoot{}
\cfoot{\small \thepage/\pageref*{LastPage}}
\rfoot{}

\fancypagestyle{firststyle}
{
\renewcommand{\headrulewidth}{0pt}%
   \fancyhf{}
   \fancyfoot[C]{\small \thepage/\pageref*{LastPage}}
}

%\def\labelitemi{--}
%\usepackage{enumitem}
%\setitemize[0]{leftmargin=25pt}
%\setenumerate[0]{leftmargin=25pt}




\makeatletter
\@ifpackageloaded{hyperref}{}{%
\ifxetex
  \usepackage[setpagesize=false, % page size defined by xetex
              unicode=false, % unicode breaks when used with xetex
              xetex]{hyperref}
\else
  \usepackage[unicode=true]{hyperref}
\fi
}
\@ifpackageloaded{color}{
    \PassOptionsToPackage{usenames,dvipsnames}{color}
}{%
    \usepackage[usenames,dvipsnames]{color}
}
\makeatother
\hypersetup{breaklinks=true,
            bookmarks=true,
            pdfauthor={ ()},
             pdfkeywords = {},  
            pdftitle={Economia Política e Instituições},
            colorlinks=true,
            citecolor=blue,
            urlcolor=blue,
            linkcolor=magenta,
            pdfborder={0 0 0}}
\urlstyle{same}  % don't use monospace font for urls


\setcounter{secnumdepth}{0}





\usepackage{setspace}

\title{Economia Política e Instituições}
\author{Danilo Freire}
\date{Março-Maio de 2017}


\begin{document}  

		\maketitle
		
	
		\thispagestyle{firststyle}

%	\thispagestyle{empty}


	\noindent \begin{tabular*}{\textwidth}{ @{\extracolsep{\fill}} lr @{\extracolsep{\fill}}}


E-mail: \texttt{\href{mailto:danilofreire@gmail.com}{\nolinkurl{danilofreire@gmail.com}}} & Web: \href{http://danilofreire.com}{\tt danilofreire.com}\\
Horário de Atendimento: A qualquer hora  &  Horário das Aulas: 12:30-15:20\\
Contato: Via e-mail  & Classe: Online\\
	&  \\
	\hline
	\end{tabular*}
	
\vspace{2mm}
	


\section{Descrição}\label{descricao}

A matéria almeja um aprofundamento sobre o conceito de instituições, o
papel que elas cumprem na coordenação social e o sentido em que conduzem
os incentivos da conduta econômica. O curso terá 18 sessões, sempre às
quartas-feira, exceto as duas primeiras sessões (quinta-feira).
Lembrando que haverá um recesso durante a Semana Santa (10 a 14/04).

\section{Objetivos}\label{objetivos}

\begin{itemize}
\tightlist
\item
  Entender o que são instituições e como normas permitem a coordenação
  de ações individuais
\item
  Trabalhar com facilidade os conceitos políticos e econômicos
  necessários para o estudo do papel das instituições
\item
  Adquirir uma visão de conjunto desta área da economia política
\item
  Conhecer e aplicar os conceitos de normas, incentivos e mudanças
  institucionais que fundamentam a relação entre mercado e estado
\item
  Familiarizar-se com a história intelectual das instituições
  político-econômicas e como a área se relaciona com outras disciplinas
\end{itemize}

\newpage

\section{Cronograma e Bibliografia}\label{cronograma-e-bibliografia}

Todas as referências de leitura estão disponíveis no
\href{http://github.com/danilofreire/eei-omma-ufm}{repositório do curso
no GitHub}. Recomenda-se que os alunos leiam os textos antes das sessões
e que tragam pontos a serem discutidos com o tutor e demais colegas. A
participação em classe é fundamental para que as aulas sejam dinâmicas e
interessantes.

\subsection{Sessões 1 e 2: A Ordem Institucional
(23/03)}\label{sessoes-1-e-2-a-ordem-institucional-2303}

Programa do curso. Introdução ao conceito de instituição dentro da
economia política. Considerações sobre as origens evolutivas das
instituições como resultado de ações humanas propositivas, mas não do
desígnio humano. Apanhado da história intelectual de instituições dentro
do conceito de ordem.

\textbf{Leituras}:

\begin{itemize}
\tightlist
\item
  Norman Barry - The Tradition of Spontaneous Order
\item
  Geoffrey M. Hodgson - What Are Institutions?
\end{itemize}

\subsection{Sessões 3 e 4: Diversidade Institucional I: Policentrismo
(29/03)}\label{sessoes-3-e-4-diversidade-institucional-i-policentrismo-2903}

Considerações acerca dos modelos de administração pública. Introdução
aos conceitos de mono e policentrismo institucional. Discussão sobre
vantagens e desafios de arranjos policêntricos. Federalismo e
multiplicidade de jurisdições.

\textbf{Leituras}:

\begin{itemize}
\tightlist
\item
  Vincent Ostrom - The Intellectual Crisis in American Public
  Administration. Capítulo 1: The Crisis of Confidence e Capítulo 2: The
  Intellectual Mainstream in American Public Administration.
\item
  Vincent Ostrom - Polycentricity: The Structural Basis of
  Self-Governing Systems
\item
  Paul Dragos Aligica - Institutional Diversity and Political Economy:
  The Ostroms and Beyond. Capítulo 2: Institucionalism and
  Polycentricity.
\end{itemize}

\subsection{Sessões 3 e 4: Diversidade Institucional I: Policentrismo
(29/03)}\label{sessoes-3-e-4-diversidade-institucional-i-policentrismo-2903-1}

Considerações acerca dos modelos de administração pública. Introdução
aos conceitos de mono e policentrismo institucional. Discussão sobre
vantagens e desafios de arranjos policêntricos. Federalismo e
multiplicidade de jurisdições.

\textbf{Leituras}:

\begin{itemize}
\tightlist
\item
  Vincent Ostrom - The Intellectual Crisis in American Public
  Administration. Capítulo 1: The Crisis of Confidence e Capítulo 2: The
  Intellectual Mainstream in American Public Administration.
\item
  Vincent Ostrom - Polycentricity: The Structural Basis of
  Self-Governing Systems
\item
  Paul Dragos Aligica - Institutional Diversity and Political Economy:
  The Ostroms and Beyond. Capítulo 2: Institucionalism and
  Polycentricity.
\end{itemize}

\subsection{Sessões 5 e 6: Diversidade Institucional II: IAD Framework
(05/04)}\label{sessoes-5-e-6-diversidade-institucional-ii-iad-framework-0504}

Considerações sobre o trabalho de Elinor Ostrom. Introdução ao Framework
de Análise e Desenvolvimento Institutional. Aplicações teóricas e
práticas.

\textbf{Leituras}:

\begin{itemize}
\tightlist
\item
  Elinor Ostrom - The Three Worlds of Action
\item
  Elinor Ostrom - An Agenda for the Study of Institutions
\end{itemize}

\subsection{Sessões 7 e 8: Instituições no Contractarianismo
(19/04)}\label{sessoes-7-e-8-instituicoes-no-contractarianismo-1904}

Como uma teoria institucional pode derivar autoridade e legitimidade a
partir do consentimento dos governados. A forma e o conteúdo do
consentimento e a ideia de contrato e de acordos mutuamente benéficos. A
origem da força normativa e da possibilidade de reconhecimento e
execução de instituições. A possibilidade de se modelar o comportamento
econômico. A aplicação de modelos econômicos na interpretação
institucional.

\textbf{Leituras}:

\begin{itemize}
\tightlist
\item
  James Buchanan \& Geoffrey Brennan - The Reason of Rules:
  Constitutional Political Economy. Capítulos 2 a 5.
\end{itemize}

\subsection{Sessões 9 e 10: Instituições no Ocidente, Jusnaturalismo e
Utilitarismo
(26/04)}\label{sessoes-9-e-10-instituicoes-no-ocidente-jusnaturalismo-e-utilitarismo-2604}

A possibilidade de um sistema institucional dedutivo. A capacidade de
derivar ordem normativa da razão humana e do princípio da utilidade. O
direito positivo como construção verbal a partir da convergência de
interesses humanos. Uma breve história das origens concretas do
estabelecimento institucional no ocidente.

\textbf{Leituras}:

\begin{itemize}
\tightlist
\item
  Harold Berman - Law and Revolution (extratos)
\item
  David Gauthier - Morals by Agreement, Capítulo 1.
\end{itemize}

\subsection{Sessões 11 e 12: Instituições no Terceiro Mundo e o Caso
Islâmico
(03/05)}\label{sessoes-11-e-12-instituicoes-no-terceiro-mundo-e-o-caso-islamico-0305}

Relação entre a teoria do capital e a teoria institucional. O papel das
instituições no desenvolvimento econômico histórico. Como ocorre a
destruição criadora em sociedades em desenvolvimento e de que maneira a
coordenação econômica possibilitada pelas instituições fortalece ou
previne o crescimento econômico. Estudo da emergência de instituições
econômicas no mundo islâmico e seus efeitos no desenvolvimento regional.

\textbf{Leituras}:

\begin{itemize}
\tightlist
\item
  Hernando de Soto - O Mistério do Capital (Capítulos 1-3)
\item
  Timur Kuran -
  \href{https://github.com/danilofreire/eei-omma-ufm/blob/master/sessoes-11-12/kuran-01.pdf}{The
  Puzzle of the Middle East's Economic Underdevelopment}
\item
  Timur Kuran -
  \href{https://github.com/danilofreire/eei-omma-ufm/blob/master/sessoes-11-12/kuran-02.pdf}{The
  Economic Impact of Islamism}
\end{itemize}

\subsection{Sessões 13 e 14: O Desenho das Instituições
(10/05)}\label{sessoes-13-e-14-o-desenho-das-instituicoes-1005}

Bens públicos e redistribuição econômica. A motivação de governantes e
governantes na construção institucional. Agregado e distribuição de
preferências. A possibilidade de se planejar coordenação econômica a
partir de um desenho institucional. A base institucional das regulações
econômicas. Comparação entre regulações privadas e econômicas. Aplicação
da racionalidade econômica na aplicado de políticas públicas.

\textbf{Leituras}:

\begin{itemize}
\tightlist
\item
  Anthony Ogus -
  \href{https://github.com/danilofreire/eei-omma-ufm/blob/master/sessoes-13-14/ogus.pdf}{Economics
  and the Design of Regulatory Law (Capítulo 5)}
\item
  Mancur Olson -
  \href{https://github.com/danilofreire/eei-omma-ufm/blob/master/sessoes-13-14/olson.pdf}{Dictatorship,
  Democracy, and Development}
\end{itemize}

\subsection{Sessões 15 e 16: Instituições Além do Estado I: Comércio
(17/05)}\label{sessoes-15-e-16-instituicoes-alem-do-estado-i-comercio-1705}

Como instituições produzidos de forma privada e regulamentos econômicos
privados podem ser exequíveis e eficazes. Casos históricos de
instituições de governança privada demonstrando sua emergência e
explicando seu funcionamento.

\textbf{Leituras}:

\begin{itemize}
\tightlist
\item
  Edward Stringham -
  \href{https://github.com/danilofreire/eei-omma-ufm/blob/master/sessoes-15-16/stringham-01.pdf}{The
  Extralegal Development of Securities Trading in seventeenth-century
  Amsterdam}
\item
  Edward Stringham -
  \href{https://github.com/danilofreire/eei-omma-ufm/blob/master/sessoes-15-16/stringham-02.pdf}{The
  Alternative of Private Reguation: The London Stock Exchange's
  Alternative Investment Market as a Model}
\end{itemize}

\subsection{Sessões 17 e 18: Instituições Além do Estado II:
Organizações Ilegais
(24/05)}\label{sessoes-17-e-18-instituicoes-alem-do-estado-ii-organizacoes-ilegais-2405}

Como instituições produzidos de forma privada e regulamentos econômicos
privados podem ser exequíveis e eficazes. Casos históricos de
instituições de governança privada demonstrando sua emergência e
explicando seu funcionamento.

\textbf{Leituras}:

\begin{itemize}
\tightlist
\item
  David Skarbek -
  \href{https://github.com/danilofreire/eei-omma-ufm/blob/master/sessoes-17-18/skarbek.pdf}{Covenants
  without the Sword? Comparing Prison Self-Governance Globally}
\item
  Danilo Freire -
  \href{https://github.com/danilofreire/eei-omma-ufm/blob/master/sessoes-17-18/freire.pdf}{Beasts
  of Prey or Rational Animals? Private Governance in Brazil's Jogo do
  Bicho}
\end{itemize}




\end{document}

\makeatletter
\def\@maketitle{%
  \newpage
%  \null
%  \vskip 2em%
%  \begin{center}%
  \let \footnote \thanks
    {\fontsize{18}{20}\selectfont\raggedright  \setlength{\parindent}{0pt} \@title \par}%
}
%\fi
\makeatother
